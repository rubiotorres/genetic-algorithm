% -----------------------------------------------------------------------------
%   Arquivo: ./01-texto/resumo.tex
% -----------------------------------------------------------------------------

\resumoA{
Síntese do trabalho em texto cursivo contendo um único parágrafo com, no máximo, 200 palavras. O resumo é a apresentação clara, concisa e seletiva do trabalho. No resumo deve-se incluir, preferencialmente, nesta ordem: brevíssima introdução ao assunto do trabalho de pesquisa (incluindo motivação e justificativa para a realização deste trabalho), o que será feito no trabalho (objetivos), como ele será desenvolvido (metodologia), quais são os principais resultados obtidos ou esperados e a conclusão (compare os resultados com os da literatura e destaque as principais contribuições científicas do trabalho.
}


\palavrachaveA{
Modelo. Artigo científico. Redação técnica. Outra palavra.
}


% -----------------------------------------------------------------------------
%   Escolha de 3 a 6 palavras ou termos que descrevam este trabalho. As palavras-chaves são utilizadas para indexação.
%   A letra inicial de cada palavra deve estar em maiúsculas. As palavras-chave são separadas por ponto.
% -----------------------------------------------------------------------------
